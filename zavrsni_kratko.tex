\documentclass[a4paper,12pt]{article}
\usepackage[utf8x]{inputenc}
\usepackage[croatian]{babel}
\usepackage[T1]{fontenc}
\usepackage{hyperref}

\usepackage[top=2.5cm, bottom=2.5cm, left=2.5cm, right=2.5cm]{geometry}
%\usepackage{setspace}
\usepackage{parskip} %razmak izmedju paragrafa i uvlacenje prvog retka
\usepackage{mathpazo}

\usepackage{tabularx}


\title{Kratke upute za izradu završnog rada}
\author{Ljiljana Despalatović}
\date{}
\begin{document}

\maketitle

\section{Version control system}
Pri izradi završnog rada nužno je koristiti neki od sustava za kontrolu verzija. Ideja je da se omogući praćenje promjena u programskom k\^odu i u dokumentaciji uz komentiranje, što omogućava povratak na prijašnje verzije. Neki od sustava za kontrolu verzija su:  
\begin{itemize}
 \item git
 \item svn 
 \item mercurial
 \item cvs
\end{itemize}

Odabrani sustav trebate lokalno instalirati, a onda dodati postojeće ili novokreirane datoteke vezane uz projekt u lokalni repozitorij. Raditi sa sustavom za kontrolu verzija možete iz komandne linije ili pomoću grafičkih klijenata (Gitextensions, SmartGit-Hg, Giggle, TortoiseGit).

Osim praćenja promjena, ideja \textit{version control} sustava je kolaboracija. S obzirom da na završnom radu radite u kolaboraciji sa mentorom tj.~sa mnom, projekt bi se trebao nalaziti na nekom od centralnih web repozitorija. Najpopularniji su:
\begin{itemize}
 \item GitHub (git, svn)
 \item Bitbucket (git, mercurial)
 \item GitLab (git)
 \item GNU Savannah (cvs, git, mercurial, svn) - samo za free softver projekte
 \item SourceForge (cvs, git, mercurial, svn) - samo za open source projekte
\end{itemize}

Preporučena kombinacija je git/GitHub ili git/Bitbucket, ali mogu i ostale kombinacije. Možete i sami kreirati web repozitorij na nekom serveru kojem imate pristup (primjer \href{https://civicactions.com/blog/2010/may/25/how\_set\_svn\_repository\_7\_simple\_steps}{uputa} za svn repozitorij).

\section{Timeline}
 \setlength{\extrarowheight}{.5em}
\begin{tabularx}{\linewidth}{Xr}
\textsc{vremenski okvir}& \textsc{akcija}\\
\hline
svaki mjesec od 1.~do 8.~u mjesecu, u ispitnom roku prva tri tjedna roka & prijava obrane\\
dva tjedna prije prijave & gotov rad na uvid komisiji\\
tri tjedna prije prijave & gotov rad na uvid mentoru\\
\hline
\end{tabularx}
\end{document}
