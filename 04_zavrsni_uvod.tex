\section{Uvod}
Prilikom pisanja seminarskih i završnih radova, studenti se suočavaju sa problemom "kako pisati". Na žalost, tijekom školovanja rijetko su u prilici pisati seminarske radove pa tako nemaju dobru praksu. Veliki problem predstavlja nepostojanje materijala kojima bi ih se uputilo u osnove pravilnog pisanja radova. Namjena ovog teksta 
je obuhvatiti na jednom 
mjestu osnovne smjernice za pisanje završnog rada.

Sam postupak opisan je u Pravilniku za izradu i obranu završnih radova koji je dostupan na našim internetskim stranicama.
%Ove upute su sažetak tog pravilnika s nekim dodatnim napomenama.

%Nekoliko stvari utječe na kvalitetu rada. Prvi dojam pri čitanju dobije se na temelju izgleda dokumenta. Važno je rad korektno formatirati, odabrati fontove, veličine slova i prorede koji su ugodni za čitanje i ne odvraćaju pozornost čitatelja sa teme. Studenti su dužni pisati pravopisno i gramatički ispravne i smislene rečenice. Radove koji nisu pravopisno i gramatički ispravno napisani, mentor neće prihvatiti.

%Najvažnije pravilo kojega se studenti moraju pridržavati prilikom izrade rada je izbjegavanje plagijata. Zabranjeno je doslovno preuzimati tuđe rečenice i dijelove teksta. Ukoliko se želi citirati nekog drugog autora, potrebno je navesti izvor citata i to u obliku fusnote ili sa oznakom koja upućuje na popis literature\nocite{*}. Doslovni prijevod teksta sa nekog drugog jezika je također plagijat.

U ovom tekstu navode se osnovne smjernice za izradu završnog rada. Osim toga, ovaj dokument može poslužiti kao \LaTeX \ predložak za pisanje završnog rada. Svi izrazi koji se koriste u tekstu, a imaju rodno značenje, obuhvaćaju na jednak način sve rodove. 

U uvodu se opisuje problem koji se rješava u završnom radu, motivacija za odabir problema, kratki opis načina na koji je problem riješen te glavni doprinos rješenja s obzirom na dani problem. Na kraju uvoda daje se sažeti opis ostalih poglavlja rada. 

Nakon uvoda, u drugom poglavlju dan je postupak prijave i izrade završnog rada. U trećem 
poglavlju je opisana struktura završnog rada i dane su smjernice za oblikovanje rada. Na kraju, u zadnjem poglavlju je dan zaključak.


 