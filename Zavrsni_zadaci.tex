\section{Izrada praktičnog dijela završnog rada}
\subsection{Općenito o izradi praktičnog dijela završnog rada}
Temu za izradu završnog rada studenti biraju između ponuđenih tema ili sami predlažu teme. Programski alati i tehnologije ne smiju ovisiti 
o operativnom sustavu i verzijama operativnog sustava i alata (uz dogovorene iznimke). Uz sam praktični rad predaje se i uputa za pokretanje 
projekta (tipično u README.txt datoteci).

Primjerice, programi izrađeni u C, C++ i Javi trebaju koristiti standardne
 biblioteke zajedničke za sve platforme i sl.
 
 K\^od praktičnog dijela završnog rada (a poželjno je i pismeni dio završnog rada) treba biti pohranjen na nekom repozitoriju koji omogućava rad sa nekim od \textit{version control} sustava (npr. git, cvs, subversion). Mentor treba imati pristup repozitoriju.
Na taj način sva povijest izrade projekta ostaje zabilježena, te je olakšano praćenje projekta i povratak na neku stariju/ispravnu verziju. Ukoliko pisani dio nije verzioniran putem \textit{version control}
 sustava, iz naziva poslanih (na čitanje) datoteka treba biti jasna njihova pozicija u vremenu.  
 
Repozitorije možete kreirati na \url{github.com}, \url{bitbucket.org} ili nekom trećem. Za rad za \textit{version control} sustavom možete koristiti neki od GUI klijenata za rad sa 
 \texttt{git}-om (\url{git-scm.com/downloads/guis}), \texttt{subversion} sustavom (\url{http://svn-ref.assembla.com/windows-svn-client-reviews.html}) ili raditi iz komandne linije.
\subsection{Teme za završni rad}
Teme za završni rad definiraju se na početku akademske godine. Studenti mogu odabrati jednu od ponuđenih tema ili ponuditi svoju. Kao mentor prihvaćam i teme koje nisu usko vezane uz tehnologije o kojima predajem, ali tada je moja podrška ograničena.

Literatura u popisu tema nije konačna. Prilikom odabira teme studenti će dobiti daljnju literaturu.

Popis tema objavljen ja na \textit{moodle.oss.unist.hr} na forumu kolegija Završni rad.








