\section{Uvod}
Prilikom pisanja seminarskih i završnih radova, studenti se suočavaju sa problemom "kako pisati". Na žalost, tijekom školovanja rijetko su u prilici 
pisati seminarske radove, pa tako 
nemaju dobru praksu. Veliki problem predstavlja nepostojanje materijala kojima bi ih se uputilo u osnove pravilnog pisanja radova. Namjena ovog teksta 
je obuhvatiti na jednom 
mjestu osnovne smjernice za pisanje završnog rada.

Nekoliko stvari utječe na kvalitetu rada. Prvi dojam pri čitanju dobije se na temelju izgleda dokumenta. Važno je rad korektno formatirati, odabrati fontove, 
veličine slova i prorede 
koji su ugodni za čitanje i ne odvraćaju pozornost čitatelja sa teme. Studenti su dužni pisati pravopisno i gramatički ispravne i smislene rečenice. 
Radove koji nisu pravopisno i gramatički ispravno napisani, mentor neće prihvatiti.

Najvažnije pravilo kojega se studenti moraju pridržavati prilikom izrade rada je izbjegavanje plagijata. Zabranjeno je doslovno preuzimati tuđe rečenice 
i dijelove teksta. Ukoliko 
se želi citirati nekog drugog autora, potrebno je navesti izvor citata i to u obliku fusnote ili sa oznakom koja upućuje na popis literature\nocite{*}. 
Doslovni prijevod teksta sa nekog drugog 
jezika je također plagijat.

U ovom tekstu navode se osnovne smjernice za izradu završnog rada. Osim toga, ovaj dokument može poslužiti kao \LaTeX predložak za pisanje završnog rada. 

Nakon uvoda, u drugom poglavlju dana su pravila kako treba izgledati završni rad. 
U trećem 
poglavlju je opisana struktura završnog rada. Kako je završni rad i praktičan rad, u četvrtom poglavlju je opis izrade praktičnog dijela. U petom poglavlju 
dane su osnove komunikacije sa mentorom. Nadalje, u 
šestom poglavlju su dane upute za izradu prezentacije. Na kraju, u zadnjem poglavlju je dan zaključak.


 