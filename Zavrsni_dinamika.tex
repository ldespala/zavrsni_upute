\section{Postupak prijave i izrade završnog rada}
Završni rad na preddiplomskom studiju IT-a vrednovan je s 12 ECTS bodova što predstavlja 360 sati rada, a na specijalističkom 20 ETCS bodova (600 sati rada). %Na preddiplomskom studiju je uglavnom predviđeno da student za praktični dio rada (analiza, prikupljanje zahtjeva, teoretsko istraživanje, izrada aplikacije) iskoristi 300 sati, a za samo pisanje i pripremu obrane preostalih 60 sati. Na specijalističkom studiju IT-a rad je vrednovan s 20 ECTS bodova što predstavlja 600 sati rada, tj.~predviđeno je da student za praktični dio rada (analiza, prikupljanje zahtjeva, teoretsko istraživanje, izrada aplikacije) iskoristi približno 420 sati, a za samo pisanje i pripremu obrane preostalih 180 sati.

Priprema završnog rada započinje odabirom mentora i teme, nastavlja se izradom rada u suradnji s mentorom i završava se obranom pred povjerenstvom. Uspješno obranjen završni rad je formalni završetak školovanja na studiju čime se službeno stječu sva prava i obveze stručnog prvostupnika inženjera IT-a odnosno stručnog specijalista inženjera IT-a. 

\subsection{Odabir mentora i teme}
Izrada završnog rada je sastavni dio zadnjeg semestra studija. Da bi prijavili rad za obranu potrebno je prethodno položiti sve predmete upisane po studijskom programu i uspješno odraditi stručnu praksu. %Zatražite od Pročelnika Odsjeka za IT da vam dodijeli pristupne podatke za sustav Mentor (\url{mentor.oss.unist.hr}). Kroz sustav Mentor svi studenti IT-a moraju odabrati svog mentora te odraditi cijeli postupak izrade završnog rada. 


Mentora se odabire obzirom na područje studija koje vas najviše zanima. S njim možete dogovoriti i neku temu koja još nije objavljena pazeći pritom na složenost zadatka koja mora biti u skladu s navedenim ECTS bodovima. Tema se definira barem tri mjeseca prije planirane obrane. U suradnji s mentorom jasno odredite ciljeve izrade završnog rada. Sve teme mora odobriti kolegij Odsjeka za IT prije nego se službeno odobre studentima.

U postupku odabira mentora treba uzeti u obzir da pojedini nastavnik nije obvezan prihvatiti mentorstvo, a svaki student može od pročelnika odsjeka zatražiti da mu se dodijeli mentor.

\subsection{Izrada rada}
Preporučljivo je s izradom završnog rada započeti u zadnjem semestru studija, uz paralelno uspješno izvršavanje obaveza koje su zadane na predmetima u tom semestru.

Završni rad mora biti samostalno izrađen pod vodstvom i u suradnji s mentorom. Za sadržaj rada je odgovoran isključivo student, a mentor mu u dobroj vjeri pomaže i savjetuje ga pri samoj izradi.

Svaki pokušaj plagiranja će biti najstrože sankcioniran.

Mentor može uvjetovati obavezu korištenja nekih od alata za kolaboraciju ili verzioniranje programskog k\^oda. U tom slučaju k\^od praktičnog dijela završnog rada (a poželjno je i pismeni dio) mora biti pohranjen na nekom repozitoriju koji omogućava rad sa nekim od \textit{version control} sustava (npr. git, cvs, subversion). Mentor mora imati pristup repozitoriju. Na taj način sva povijest izrade projekta ostaje zabilježena što olakšava praćenje projekta i povratak na neku stariju/ispravnu verziju. Ukoliko pisani dio nije verzioniran putem \textit{version control} sustava, iz naziva poslanih (na čitanje) datoteka treba biti jasna njihova pozicija u vremenu.  
 
Repozitorije možete kreirati na servisima \url{github.com}, \url{bitbucket.org} ili nekom trećem. Za rad za \textit{version control} sustavom možete koristiti neki od GUI klijenata za rad sa 
 \texttt{git}-om (\url{git-scm.com/downloads/guis}), \texttt{subversion} sustavom (\url{http://svn-ref.assembla.com/windows-svn-client-reviews.html}) ili raditi iz komandne linije.
 
\subsection{Predaja rada}
Prije predaje rada student mora ispuniti sve ostale obaveze studijskog programa. Po završetku izrade rada mentor će od Pročelnika Odsjeka IT-a zatražiti formiranje Povjerenstva za završni rad. Povjerenstvo je sastavljeno od tri člana. Predsjednik povjerenstva je mentor, a uz njega su još i dva visokoškolska nastavnika. Povjerenstvo pregledava rad i ustanovljuje njegovu usklađenost s odobrenom temom te obavještava mentora o eventualnim korekcijama koje je potrebno učiniti u radu. Povjerenstvo može i odbiti rad ukoliko student nije izradio rad u skladu s temom ili je nije realizirao na zadovoljavajući način prema dogovorenim uvjetima. Postupak korekcija rada nastavlja se sve dok povjerenstvo (svi njegovi članovi) nisu prihvatili rad za predaju. 

Ukoliko je student napravio ispravke po uputama povjerenstva, mentor odobrava tiskanje dva primjerka rada na kojima će on potpisom potvrditi prikladnost za obranu. Ta dva primjerka zajedno sa dva CD-a na kojima se nalaze svi dokumenti i aplikacija predaju se u studentsku referadu. Bitno je da se na CD-u nalazi i datoteka s pisanim dijelom rada koji je u skladu sa standardom PDF/A – ISO 19005). Uz to se prilažu i Zahtjev za odobrenje teme završnog rada te Prijava za pristup obrani završnog rada koje potpisuje mentor (dokumenti se nalaze na \url{https://www.oss.unist.hr/odjel/propisi-i-dokumenti}). Time je postupak predaje rada završen.

Rokovi za predaju rada referadi su određeni Pravilnikom za izradu i obranu završnog rada. Prije toga treba završiti gore opisani postupak. Kako bi bili sigurni da će rad biti predan u željenom roku trebali bi najmanje osam radnih dana prije planirane predaje imati konačnu verziju koju će mentor poslati povjerenstvu na pregled. U načelu, to bi to značilo da je praktični dio rada (aplikacija) završen 20 dana prije planirane predaje. Predlažemo da se o dinamici izrade završnog rada dogovorite sa svojim vašim mentorom. Ukoliko se prekorače dogovoreni rokovi, rad će se moći predati u sljedećem roku za prijavu.

\subsection{Obrana rada}
Nakon uspješne predaje rada potrebno je obaviti i usmenu obranu rada. Student je dužan napisati prezentaciju kojom predstavlja svoj rad pred povjerenstvom. Povjerenstvo je sastavljeno od tri člana. Obrana rada je javna i vodi je predsjednik povjerenstva. Za prezentaciju se može koristiti računalo i projektor u učionici. Dodatna oprema može se koristiti u dogovoru s mentorom.

Izlaganje na obrani završnog rada treba biti kratko (10 do 15 minuta) i koncizno. Kao i sam rad, izlaganje treba imati strukturu: 
uvod, opis završnog rada, zaključak. Izlaganje počinje uvodom u materiju, gdje se opisuje tema i motivacija za temu, a završava zaključkom.

Izlaganje treba biti fluidno. Treba izbjegavati čitanje pripremljenog materijala. U tu svrhu dobra je preporuka nekoliko puta kući isprobati izlaganje. Pristupnik prezentira završni rad u stojećem položaju, okrenut prema povjerenstvu (a ne prema ploči).

Nakon izlaganja članovi povjerenstva postavljaju pitanja studentu, te se eventualno traži prezentacija praktičnog dijela. 

Nakon obrane student izlazi iz učionice, a povjerenstvo održi kratki sastanak na kojem se ocijenjuje rad i obrana. Pri povratku predsjednik povjerenstva objavljuje ocjene za rad, obranu te konačnu ocjenu završnog rada. 

Službena potvrda o završetku studija i stečenom nazivu može se predignuti u studentskoj referadi u pravilu sedam dana nakon obrane.