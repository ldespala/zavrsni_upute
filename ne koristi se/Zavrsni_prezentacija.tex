\section{Obrana završnog rada rada}
\subsection{Dinamika obrane završnog rada}
Izlaganje na obrani završnog rada treba biti kratko (10 do 15 minuta) i koncizno. Kao i sam rad, izlaganje treba imati strukturu: 
uvod, opis završnog rada, zaključak. Izlaganje počinje uvodom u materiju, gdje se opisuje tema i motivacija za temu, a završava zaključkom.

Izlaganje treba biti fluidno. Treba izbjegavati čitanje pripremljenog materijala. U tu svrhu dobra je preporuka nekoliko puta kući isprobati izlaganje. Pristupnik prezentira završni rad u stojećem položaju, okrenut prema komisiji (a ne prema ploči).

Nakon izlaganja komisija postavlja pitanja studentu, te eventualno traži prezentaciju praktičnog dijela. 

\subsection{Prezentacijski materijali}
Prezentacije mogu biti napisani pomoću raznih alata. Primjerice, MS Power Point, Apple Keynote, LaTeX Beamer ili Prosper, 
Google Presentations itd. Bez obzira na alat kojim se izrađuje 
prezentacija, mentoru treba biti dostavljena u pdf obliku na uvid. 

Praksa (\textit{rule of thumb}) je pokazala da se za svaki slajd prezentacije računa 3 minute izlaganja. 
Nadalje, na pojedinom slajdu ne treba biti previše informacija, jer se na taj način gubi na 
čitljivosti. Dakle, na slajdu trebaju biti samo natuknice, slike, grafikoni ili neki drugi multimedijski zapisi. 

Minimalna veličina slova je 18pt, a maksimalni broj redaka teksta 10.

Studentima su na raspolaganju projektor, ploča i računalo. Studenti mogu prezentirati završni rad i na vlastitom računalu sa hdmi sučeljem.
